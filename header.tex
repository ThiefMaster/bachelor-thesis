%
% Diplomarbeit mit Latex
% ===========================================================================
%
% Edited and enhanced headerfile by Peter Hense
% (2010)
%
\documentclass[%
    pdftex,%              PDFTex verwenden
    a4paper,%             A4 Papier
    oneside,%             Einseitig (für scrbook zweiseitig einstellen!)  beachte ihead/ohead
    headsepline,%         Linie nach Kopfzeile
    footsepline,%         Linie vor Fusszeile
    12pt,%                Größere Schrift, besser lesbar am bildschirm
    %titlepage            starts a new page for a new document title. Default in scrbook, but not in article
]{scrbook}%               scrartcl für ohne Chapters, sonst scrreprt

%%%%%%%%%%%%%%%%%%%%%%%%%%%%%%%%%%%%%%%%%%%%%%%%%%%%%%%%%%
%% Pagedesign
%%
\usepackage{geometry}
%% Layoutanforderung:
\geometry{verbose,a4paper,tmargin=3cm,bmargin=3cm,lmargin=4cm,rmargin=2cm}
\usepackage{lscape}
%% leere letzte Seite
\usepackage{lastpage}
%% Element fest positionieren
\usepackage{float}
\usepackage{floatflt}
\usepackage{nonfloat}
%% Befehl, damit alle float-Elemente bis zu dem Punkt positioniert werden -> \FloatBarrier
\usepackage{placeins}

%
% Abstände zwischen Abschnitten und Zeilen
%
\setlength{\parindent}{0ex}                                                 % Einzug erste Zeile neues Kapitel
\setlength{\parskip}{2.0ex plus 0.9ex minus 0.4ex}     % Lücke zwischen zwei Absätzen
%
% Zeilenabstand auf 1,5 Zeilen (Anforderung Diplomarbeiten)
%
\usepackage{setspace}%
\onehalfspacing


%%%%%%%%%%%%%%%%%%%%%%%%%%%%%%%%%%%%%%%
% Inhaltsverzeichnis und Titelseite   %
%%%%%%%%%%%%%%%%%%%%%%%%%%%%%%%%%%%%%%%
\usepackage{titletoc}

%% Index, TOC, Bibliography mit ins Inhaltsverzeichnis aufnehmen
%\usepackage[nottoc]{tocbibind}

%% Ebenen des Toc neu definieren (kann Ebenen tiefer anzeigen.
%% Ab \chapter 3 Ebenen tiefer
\setcounter{tocdepth}{3}
\setcounter{secnumdepth}{3}

%%%%%%%%%%%%%%
% Fußnoten   %
%%%%%%%%%%%%%%
%
%% Speichert Fußnoten
%
\usepackage{savefnmark}


%%%%%%%%%%%%%%%%%%%%%%%%%
% Kopf und Fußzeilen    %
%%%%%%%%%%%%%%%%%%%%%%%%%
\usepackage[automark,headsepline,nouppercase]{scrpage2}
%% Abstand der Kapitelüberschrift von der Kopfzeile
\renewcommand*{\chapterheadstartvskip}{\vspace*{-\topskip}}
%% Auch bei Kapitelüberschriften Kopfzeile anwenden
\renewcommand*{\chapterpagestyle}{scrheadings}
\renewcommand*{\indexpagestyle}{scrheadings}

%
% Kopf und Fußzeilen anpassen
%
%% Komascript Kopfzeilen
\pagestyle{scrheadings}
%% Kapitelüberschriften in den headern automatisch aktualisieren.
\automark[chapter]{chapter}
%% Felder zurücksetzen
\clearscrheadfoot
%% linke Kopfzeile (für zweiseitig -> scrbook)
%\ihead[\headmark]{\headmark}
%% rechte Kopfzeile (einseitige Dokumente)
%% durch Angabe des optionalen Parameters ändert man scrplain mit!
\ohead[\headmark]{\headmark}
%% rechte Fußzeile: Seitenzahl (\thepage)
\ofoot[\pagemark]{\pagemark}
%% Kopfzeilentext + Linie: Verschiebung vom linken Rand weg, Breite der Kopfzeile
\setheadwidth[0pt]{text}
%% Stärke der Trennlinie zum Text
\setheadsepline{.4pt}

%
% alternative: sepline centered:
%
% \usepackage[footsepline, footbotline]{scrpage2}
% \setfootbotline{2pt}
% \setfootsepline[text]{.4pt}
% \setfootwidth[0pt]{textwithmarginpar}


%%%%%%%%%%%%%%%%%%%%
% Schriftarten     %
%%%%%%%%%%%%%%%%%%%%

% Schriften
\usepackage[utf8]{inputenc}
\usepackage[T1]{fontenc}
\usepackage{ae}
\usepackage{verbatim}

%% Verwendung von "`Times"'
%\usepackage{times}
%\usepackage{mathptmx}           % Times + passende Mathefonts


%% Verwendung von "`Garamond"'
%\renewcommand{\rmdefault}{ugm}
%\usepackage[urw-garamond]{mathdesign}

%
% Paket für Darstellung im Deutschen
%
\usepackage[ngerman]{babel}
\usepackage[babel,german=quotes]{csquotes}

%
% Type 1 Fonts für bessere darstellung in PDF verwenden.
%
%\usepackage[scaled=.92]{helvet} % skalierte Helvetica als \sfdefault
\usepackage{courier}            % Courier als \ttdefault

%
% Spezielle Schrift verwenden.
%
%\renewcommand{\encodingdefault}{T1}

%
% Spezielle Schrift im Koma-Script setzen.
%
\setkomafont{sectioning}{\normalfont\bfseries}
\setkomafont{captionlabel}{\normalfont\bfseries}
\setkomafont{pageheadfoot}{\normalfont\itshape}
\setkomafont{descriptionlabel}{\normalfont\bfseries}

%%%%%%%%%%%%%%%%%%%%%%%%%%%%%%%%%%%%%%%%%%%%%%%
% Tabellenüber- und Abbildungsunterschriften  %
%%%%%%%%%%%%%%%%%%%%%%%%%%%%%%%%%%%%%%%%%%%%%%%
%\usepackage[font=small, labelfont=bf, format=hang]{caption}
%\captionsetup[table]{position=above, belowskip=5pt, aboveskip=5pt}
%\captionsetup[figure]{position=below, belowskip=5pt, aboveskip=5pt}
\usepackage{tabularx}


%%%%%%%%%%%%%%%%%%%%%%%%%%%%%%%%%%%%%%%%%%%%%%%%%
% PDF Generierung                               %
%%%%%%%%%%%%%%%%%%%%%%%%%%%%%%%%%%%%%%%%%%%%%%%%%
\usepackage{pdfpages}
\usepackage{pst-pdf}
\usepackage{pstricks, pst-node, pst-plot, pst-math, pst-optic, pst-lens, pst-text, pst-xkey, pstricks-add}
\usepackage{ifpdf}
\ifpdf
        \usepackage[
            pdftex=true,
            hyperfigures=true,
            backref=false,
            hyperindex=true,
            bookmarksnumbered=true,
            bookmarksopen=true,
            colorlinks=true,
            citecolor=black,
            linkcolor=black,
            urlcolor=black,
            filecolor=magenta,
            pdfborder={0 0 0},
            plainpages=false
            ]{hyperref}
        \pdfinfo{
            /Title (Bachelor-Thesis)
            /Subject (Redesign und Modernisierung von Indicos JavaScript-Framework)
            /Author (Adrian Mönnich)
            /Keywords (JavaScript Indico CERN Framework JQuery)
        }
        \else
            \usepackage[
                hypertex=true,
                hyperref=true,
                backref=true,
                hyperindex=true,
                bookmarksnumbered=true,
                colorlinks=true,
                inkcolor=blue,
                urlcolor=blue,
                filecolor=magenta,
                german
            ]{hyperref}
\fi

%% Gesataltung von Hyperlinks
\urlstyle{sc}

%%%%%%%%%%%%%%%%%%%%%%%%%%%%
%% Spezielle Ergänzungen für PS
%%%%%%%%%%%%%%%%%%%%%%%%%%%%

%% Farben für PSforPDF definieren
\usepackage{color}

%%%%%%%%%%%%%%%%%%%%%%%%%%%%%%%%%%%%%%%%%%%%%%%%%%%%%%%%%
% MISC                                                  %
%-------------------------------------------------------%
%%%%%%%%%%%%%%%%%%%%%%%%%%%%%%%%%%%%%%%%%%%%%%%%%%%%%%%%%
%
% Paket für die Indexerstellung.
%
\usepackage{makeidx}

%
% Dictum-Breite vergrößern
%
\renewcommand*{\dictumwidth}{.6667\textwidth}


%
% Paket um Grafiken einbetten zu können
%
\usepackage{graphics}
\usepackage[pdftex]{graphicx}       %% add PNG import capabilities
\DeclareGraphicsRule{*}{mps}{*}{}

%
% mathematische symbole aus dem AMS Paket.
%
\usepackage{amsmath, amsthm, amssymb}
%\usepackage{amssymb} % already included if you use a special math-font

%
% Paket um Textteile drehen zu können %
%
\usepackage{rotating}

%
% Befehle werden wiederholbar
%
\usepackage{multido}

%%%%%%%%%%%%%%%%%%%%%%%%%%%%%%%%%%%%%%%%%%%%%%
% Paket um Listings sauber zu formatieren    %
%%%%%%%%%%%%%%%%%%%%%%%%%%%%%%%%%%%%%%%%%%%%%%
\usepackage[savemem]{listings}
\definecolor{lightgray}{rgb}{0.98, 0.98, 0.98}
\definecolor{darkgray}{rgb}{0.4, 0.4, 0.4}
\definecolor{purple}{rgb}{0.65, 0.12, 0.82}
\definecolor{darkgreen}{cmyk}{0.7, 0, 1, 0.5}

\lstdefinelanguage{JavaScript}{
  morekeywords={typeof, new, true, false, catch, function, return, null, catch, switch, var, if, in, for, while, do, else, case, break},
  keywordstyle=\color{blue}\bfseries,
  ndkeywords={this, self},
  ndkeywordstyle=\color{darkgray}\bfseries,
  identifierstyle=\color{black},
  sensitive=false,
  comment=[l]{//},
  morecomment=[s]{/*}{*/},
  commentstyle=\color{purple}\ttfamily,
  stringstyle=\color{red}\ttfamily,
  morestring=[b]',
  morestring=[b]"
}[keywords,comments,strings]

\lstdefinelanguage{diff}
{
    morekeywords={+, -},
    sensitive=false,
    morecomment=[l]{//},
    morecomment=[s]{/*}{*/},
    morecomment=[l][\color{darkgreen}]{+},
    morecomment=[l][\color{red}]{-},
    morestring=[b]",
}


\lstset{
  language=JavaScript,                % choose the language of the code
  basicstyle=\ttfamily,       % the size of the fonts that are used for the code
  numbers=left,                   % where to put the line-numbers
  numberstyle=\small,      % the size of the fonts that are used for the line-numbers
  numbersep=5pt,                  % how far the line-numbers are from the code
  backgroundcolor=\color{lightgray} ,% choose the background color. You must add \usepackage{color}
  showspaces=false,               % show spaces adding particular underscores
  showstringspaces=false,         % underline spaces within strings
  showtabs=false,                 % show tabs within strings adding particular underscores
  frame=lines,                    % adds a frame around the code
  framesep=2mm,
  tabsize=4,                      % sets default tabsize to 2 spaces
  captionpos=b,                   % sets the caption-position to bottom
  breaklines=true,                % sets automatic line breaking
  breakatwhitespace=true,         % sets if automatic breaks should only happen at whitespace
  numberbychapter=false,
  aboveskip=0.75cm,
  belowskip=0.5cm,
  inputencoding=utf8
}


%%%%%%%%%%%%%%%%%%%%%%%%%%%%%%%%
% Erweiterungen für Tabellen   %
%%%%%%%%%%%%%%%%%%%%%%%%%%%%%%%%
%\usepackage{array}
\usepackage{longtable}
\usepackage{multirow}
\usepackage{colortab}
\usepackage{colortbl}

%%%%%%%%%%%%%%%%%%%%%%%%%%%%
% Erweiterung für Listen   %
%%%%%%%%%%%%%%%%%%%%%%%%%%%%
\usepackage{enumitem}
\usepackage{expdlist}


% Definition des Datums
\usepackage{datetime}
%\newdateformat{mydate}{\THEDAY.\THEMONTH.\THEYEAR}

%%%%%%%%%%%%%%%%%%%%%%%%%%%%%%%%%%%%%%%%
% Pakete und Einstellungen für Bibtex  %
%%%%%%%%%%%%%%%%%%%%%%%%%%%%%%%%%%%%%%%%
\usepackage{natbib}
\bibliographystyle{alpha}
%\bibpunct{[}{]}{;}{}{}{,}

\usepackage{pdfpages}

% Draft watermark
\usepackage[firstpage]{draftwatermark}
\SetWatermarkText{PREVIEW}

%%%%%%%%%%%%%%%%%%%%%%%%%%%%%%%%%%%%%%%%%%%%%%%%%%%%%%%%%%%%%%%%%%%%%%%%%%%%%%
% Neue Umgebungen etc.                                                       %
% ---------------------------------------------------------------------------%
%%%%%%%%%%%%%%%%%%%%%%%%%%%%%%%%%%%%%%%%%%%%%%%%%%%%%%%%%%%%%%%%%%%%%%%%%%%%%%

\newcommand{\todotext}[1]{
  {\color{red} TODO: #1} \normalfont
}

\addto\extrasngerman{
  \renewcommand*{\sectionautorefname}{Abschnitt}
  \renewcommand*{\subsectionautorefname}{Abschnitt}
  \renewcommand*\lstlistlistingname{Listingverzeichnis}
}

\makeatletter
\newcommand*{\textoverline}[1]{$\overline{\hbox{#1}}\m@th$}
\makeatother

\renewcommand*{\figureformat}{
  \figurename~\thefigure
}

% Hyphenation
\hyphenation{Java-Script Frame-work}

% Counters anpassen
\usepackage{chngcntr}
\counterwithout{figure}{chapter}
\counterwithout{footnote}{chapter}

% Captions abschaltbar machen
\usepackage{caption}

% Bilder
\newcommand{\img}[4]{
  \begin{figure}[!hbt]
    %\begin{center}
    \centering
      \vspace{1ex}
      \includegraphics[width=#2]{images/#1}
      \caption[#4]{\label{img.#1} #3}
    %\end{center}
    \vspace{1ex}
  \end{figure}
}

\newcommand{\imga}[3]{
  \begin{figure}[!hbt]
    %\begin{center}
    \centering
      \vspace{1ex}
      \includegraphics{images/#1}
      \caption[#3]{\label{img.#1} #2}
      \vspace{1ex}
    %\end{center}
  \end{figure}
}

\newcommand{\imgrot}[5]{
  \begin{figure}[!hbt]
    %\begin{center}
    \centering
      \vspace{1ex}
      \includegraphics[width=#2,angle=#3]{images/#1}
      \caption[#5]{\label{img.#1} #4}
    %\end{center}
    \vspace{1ex}
  \end{figure}
}

%
% EOF
%
