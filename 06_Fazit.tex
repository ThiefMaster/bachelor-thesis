\chapter{Fazit}

Im Rahmen dieser Bachelorarbeit am CERN wurden nicht nur zwei der bekanntesten JavaScript-Frameworks
unter die Lupe genommen sondern auch die \emph{Internals} des Indico-Frameworks. Bei allen
Frameworks haben sich sowohl Stärken als auch Schwächen gezeigt, wobei sich jQuery gegenüber
Prototype insbesondere durch die enthaltenen UI-Widgets behaupten konnte.

Tatsächlich auf jQuery umgestellt wurden bisher nur einige wenige Teile von Indico, wobei mit den
Dialogfenstern und Tabs Elemente ausgesucht wurden, die an möglichst vielen Stellen zum Einsatz
kommen. Im Verlauf der Arbeit hat sich neben dem Primärziel, das bestehende Framework zu ersetzen oder zu
erweitern, ein weiteres Ziel herauskristallisiert: jQuery so früh wie möglich, d.h. sobald Prototype
entfernt und Konflikte behoben waren, in die von allen Entwicklern genutzten Version zu integrieren,
sodass sowohl bei Neuentwicklungen als auch beim (nicht frameworkbezogenen) Refactoring bestehender
Elemente jQuery und jQuery-Plugins verwendet werden können.

Der Entwicklungszweig, in dem Dialoge und Tabs auf jQuery umgestellt werden, befindet sich auch nach
Abschluss dieser Arbeit noch in Entwicklung, da noch kleinere Anpassungen am Design und einige
Bugfixes notwendig sind.

Neben den bereits migrierten Bereichen von Indico gibt es noch weitere, bei denen sich ein -
zumindest DOM-Operationen betreffend - vollständiger Umstieg auf jQuery lohnen würde. Einer dieser
Bereiche ist die Validierung von Formularelementen. Dort würde sich die Nutzung des jQuery
Validation-Plugins\footnote{\href{http://bassistance.de/jquery-plugins/jquery-plugin-validation/}{http://bassistance.de/jquery-plugins/jquery-plugin-validation/}}
anbieten, wobei dabei aufgrund der dynamischen Erzeugung vieler Formulare und den teilweise
komplexen, d.h. nicht auf ein einzelnes Feld beschränkten, Validierungsregeln weitere Analysen
notwendig werden um festzustellen, ob dieses Plugin tatsächlich für den gewünschten Zweck geeignet
ist.

Langfristig bietet es sich auch an, über eine vollständige Migration nachzudenken - allerdings erst,
wenn der Anteil an jQuery-basiertem Code deutlich gestiegen ist.
