\chapter{Grundlagen}

\section{JavaScript}

\subsection{Geschichte}

JavaScript wurde 1995 von Brendan Eich für den \em{Netscape Navigator 2.0} entwickelt. Sie war auch
in allen darauf folgenden Versionen enthalten und wurde von Microsoft unter dem Namen \em{JScript}
im \em{Internet Explorer 3.0} implementiert.

Ab 1996 begann die Standardisierung der Sprache im Rahmen eines ECMA-Standards, welcher im Juni 1997
erschienen ist. Im April 1998 wurde dieser durch die ISO zum internationalen Standard \em{ISO/IEC
16262} zugelassen.

Im Laufe der Jahre wurde der ECMAScript-Standard immer erweitert und beispielsweise Reguläre
Ausdrücke, Exception-Handling und verbesserte Stringfunktionen hinzugefügt. Die aktuelle Ausgabe 5.1
des ECMAScript-Standards entspricht dem internationalen Standard \em{ISO/IEC 16262:2011}.
\citep[S. vii]{ecmascript}

Auch zum Zeitpunkt der Veröffentlichung dieser Arbeit wird der ECMAScript-Standard und damit die
Sprache JavaScript weiterentwickelt. Jeder moderne Browser unterstützt die Sprache, allerdings
nicht zwangsläufig alle Sprachelemente die in der aktuellsten Spezifikation enthalten sind.


\subsection{Anwendungsgebiete}

Der bekannteste und häufigste Verwendungszweck von JavaScript ist Web-Scripting innerhalb von
Browsern, was auch nicht weiter verwunderlich ist, da die Sprache ursprünlich für diesen Zweck
entwickelt wurde und lange Zeit ausschließlich in diesem Kontext benutzt wurde. Dabei stellt der
Webbrowser die Host-Umgebung, also die Objekte und Funktionen zu Kommunikation mit der Applikation,
in der die Scripts ausgeführt werden. Im Webbrowser sind solche Objekte beispielsweise das
Browserfenster (\lstinline{window}), Popupfenster (Rückgabewert von \lstinline{window.open()}),
die Browserhistory (\lstinline{history}), das aktuelle HTML-Dokument (\lstinline{document}).
Ebenfalls vom Browser bereitgestellt werden Objekte, die die einzelnen HTML-Elemente repräsentieren,
also beispielsweise Hpyerlinks, Formulare und Bilder. Darüberhinaus bietet der Browser Methoden, um
über verschiedene Ereignisse wie Klicks, Mausbewegungen und Änderungen an Formularfeldern
benachrichtigt zu werden.
\citep[S. 2]{ecmascript}

Ein weiteres Anwendungsgebiet ist \em{Server-Side JavaScript} (\em{SSJS}). Dabei wird JavaScript
direkt auf den (Web-)Server ausgeführt und kann beispielsweise direkt auf Datenbanken und Dateien
(des Servers) zugreifen. Während \em{SSJS} erstmalig 1996 im \em{Netscape Enterprise HTTP Server}
zum Einsatz kam, erlebte es erst fast 15 Jahre danach seinen Durchbruch. Mit
\em{node.js}\footnote{\href{http://nodejs.org}{http://nodejs.org}} existiert ein quelloffenes,
plattformunabhängiges Framework welches leichtgewichtig ist und dank einer eventbasierten Architektur
auch bei einer sehr großen Anzahl gleichzeitiger Clientverbindungen noch performant ist - eine
Eigenschaft die in klassischen prozess- oder threadbasierten Webservern wie \em{Apache} oftmals
nicht gegeben ist. Die serverseitige Host-Umgebung unterscheidet sich logischerweise stark von der
eines Browsers, da ein Webserver weder mit Fenstern noch HTML-Dokumenten arbeitet - dafür kennt er
beispielsweise HTTP-Requests und Formular\em{daten}. Die Ereignisse, auf die Scripte reagieren
können, sind dabei primär netzwerkbezogen - beispielsweise die neue Verbindung eines Clients oder
der vorzeitige Abbruch eines Seitenaufrufs. Es ist allerdings durchaus möglich, dass mit
entsprechendem Code Teile der Browserumgebung serverseitig nachgebildet werden. So ist es denkbar,
ein HTML-Dokument nicht als String darzustellen sondern als DOM-Tree. Dieser könnte dann dieselben
Methoden wie der Browser zur Verfügung stellen, sodass Operationen die sowohl clientseitig als auch
serverseitig sinnvoll sind, denselben Code nutzen können.
