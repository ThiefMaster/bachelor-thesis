\chapter{Grundlagen}

\section{HTML}

\subsection{Geschichte}
HTML wurde zusammen mit dem ersten Webbrowser und -server 1990 von Tim Berners-Lee am CERN
entwickelt, um Wissenschaftlern zu ermöglichen, digitale Dokumente auszutauschen und durch
Querverweise miteinander zu verknüpfen. Da am CERN bereits ein SGML-Dialekt für andere Zwecke
benutzt wurde, wurde HTML am ISO-Standard SGML\footnote{Standard Generalized Markup Language}
angelehnt, auf dem auch das 1998 vorgestellte XML\footnote{Extensible Markup Language} basiert.

Erstmalig durch ein RFC\footnote{Request for Comments} spezifiziert wurde HTML 1995 in der Version
2.0\footnote{\href{http://tools.ietf.org/rfc/rfc1866.txt}{http://tools.ietf.org/rfc/rfc1866.txt}}.
1997 wurde HTML 3.2 als \emph{W3C Recommendation}
veröffentlicht\footnote{\href{http://www.w3.org/TR/REC-html32}{http://www.w3.org/TR/REC-html32}}.
Nur zwei Jahre danach wurde HTML 4.01 ebenfalls als \emph{W3C Recommendation}
veröffentlicht\footnote{\href{http://www.w3.org/TR/html401/}{http://www.w3.org/TR/html401/}} und
wird auch heute noch aktiv verwendet, obwohl es bereits Nachfolger gibt.

Einer der Nachfolger von HTML 4.01 ist XHTML\footnote{eXtensible HyperText Markup Language}, eine
auf XML basierte Version von HTML. Diese Version wurde entwickelt, da XML es ermöglicht, eigene
Elemente und Attribute zu definieren und seitens der vieler Entwickler der Bedarf nach einer solchen
Erweitertungsmöglichkeit bestand. Darüber hinaus ermöglichen die Einschränkungen von XML, einfache
Parser zu entwickeln oder einen bestehenden XML-Parser zu nutzen und XHTML-Dokumente in andere
Formate zu transformieren; beispielsweise mittels XSLT\footnote{Extensible Stylesheet Language
Transformations}. \citep{w3c:xhtml}

Die neueste HTML-Version ist HTML5 und basiert auf Elementen von HTML 4.01 und XHTML 1.1.
Sie befindet sich zwar noch immer in Entwicklung, jedoch werden viele Neuerungen bereits heute von
den verbreiteten Browsern unterstützt. Neben dem Einführen neuer Features insbesondere im
Multimedia-Bereich spezifiziert HTML5 die Verarbeitung von Dokumenten mit Syntaxfehlern. \citep{w3c:html5}
Diese Dokumente wurden zuvor von fast allen Browsern angezeigt, allerdings gab es gravierende
Unterschiede zwischen den verschiedenen Browsern, sodass Seiten oftmals für einzelne Browser
optimiert wurden und in anderen Browsern nahezu unbenutzbar waren. Durch die Vorgaben von HTML5
werden Dokumente mit Syntaxfehlern zwar nicht wie in Programmiersprachen mit einer Fehlermeldugn
abgelehnt - eine radikale Lösung, die eine Vielzahl von Websites unbenutzbar machen würde - jedoch
sollen sie in jedem aktuellen Browser identisch aussehen.
Anders als HTML 4 und XHTML basiert HTML5 weder auf SGML noch auf XML, sondern definiert eigene
Parsingregeln, was unter anderem für die Verarbeitungsregeln bei Syntaxfehlern notwendig ist. Es ist
jedoch trotzdem möglich und erlaubt, XML-valides HTML5 zu schreiben, sodass man einen XML-Parser
benutzen kann. Diese Variante von HTML5 bezeichnet man auch als \emph{XHTML5}.

\img{HTML5.pdf}{100px}{Das offizielle HTML5-Logo (http://www.w3.org/html/logo/)}{HTML5-Logo}


\subsection{Elemente}
Ein HTML-Dokument setzt sich aus einigen grundlegenden Bestandteilen zusammen.

Am Anfang des Dokuments steht der \emph{Doctype}. Er teilt dem Parser und der Rendering-Engine mit,
um welche HTML-Version es sich handelt.

\begin{lstlisting}[language=HTML,caption=Doctype von HTML 4.01]
<!DOCTYPE HTML PUBLIC "-//W3C//DTD HTML 4.01 Transitional//EN" "http://www.w3.org/TR/html4/loose.dtd">
\end{lstlisting}

De Rest des Dokument ist aus \emph{Tags} aufgebaut, die wiederum \emph{Attribute} besitzen und Text
oder weitere Tags enthalten können. Auf der obersten Ebene eines HTML-Dokuments befindet sich immer
das \lstinline[language=HTML]{<html>}-Tag. Dieses enthält zwei Tags,
\lstinline[language=HTML]{<head>} und \lstinline[language=HTML]{<body>}.

Im \emph{HEAD}-Abschnitt des Dokuments finden sich neben Seitentitel und Metadaten verschiedene
Referenzen auf externe Daten, die eingebunden werden sollen. Dabei kann es sich sowohl um
Stylesheets handeln als auch um Scripts.

Der \emph{BODY}-Abschnitt enthält den eigentlichen Inhalt des Dokuments. Dieser besteht aus Text,
der durch Tags in Absätze, Tabellen, usw. unterteilt werden kann. Ebenfalls mithilfe von Tags lassen
sich Bilder, Videos und Audiodateien einbinden.

\begin{lstlisting}[language=HTML,caption=Ein HTML5-Dokument]
<!DOCTYPE HTML>
<html>
    <head>
        <title>Beispiel</title>
    </head>
    <body>
        <!-- Hier ist das eigentliche Dokument -->
        <h1>Testabschnitt</h1>
        <p>Dies ist nur ein Beispiel.</p>
        <img src="test.png" alt="Testbild">
    </body>
</html>
\end{lstlisting}

Genau wie in Programmiersprachen können auch in HTML Kommentare eingefügt werden, die vom Browser
ignoriert werden. Eine Ausnahme bilden dabei die \emph{Conditional
Comments}\footnote{\href{http://msdn.microsoft.com/en-us/library/ms537512\%28v=vs.85\%29.aspx}{http://msdn.microsoft.com/en-us/library/ms537512\%28v=vs.85\%29.aspx}}
im Microsoft Internet Explorer. Diese enthalten spezielle Scriptelemente, die den Kommentar abhängig
von der Browserversion ganz normal als Kommentar behandeln oder aber seinen Inhalt wie regulären
HTML-Code ausführen, sodass einzelne Elemente nur in bestimmten Browserversionen verarbeitet werden.
Da die Syntax so gewählt ist, dass diese Kommentare in anderen Browsern entweder ungültige Tags oder
reguläre Kommentare sind, eignen sie sich nicht nur zur Unterscheidung zwischen Internet
Explorer-Versionen sondern auch dazu, Code ausschließlich im Internet Explorer auszuführen bzw.
nicht auszuführen.


\subsection{Document Object Model (DOM)}
Das W3C\footnote{World Wide Web Consortium, \href{http://www.w3.org}{http://www.w3.org}} definiert
das DOM als \enquote{a platform- and language-neutral interface that will allow programs and scripts
to dynamically access and update the content, structure and style of [HTML] documents. The document
can be further processed and the results of that processing can be incorporated back into the
presented page.} \citep{w3c:dom-old}

Es handelt sich beim DOM also um eine standardisierte Schnittstelle, um auf die Elemente eines
HTML-Dokuments zuzugreifen und sie zu modifizieren. Das Dokument selbst wird dabei durch ein
\emph{Document}-Objekt repräsentiert, welches Methoden und Eigenschaften zum Zugriff auf die
enthaltenen Elemente enthält. Beispielsweise sucht \lstinline{getElementById()} nach
einem Element mit einer bestimmten ID und die \lstinline{documentElement}-Eigenschaft eines Elements
verweist das \lstinline[language=HTML]{<html>}-Element. Neben den Zugriffsmethoden stellt das
DOM-Interface auch Methoden zur Manipulation des Dokuments zur Verfügung. Diese reichen über das
simple Erstellen eines neuen Elements über das Entfernen eines Elements samt untergeordneten
Elementen bis zum Einfügen eines Elements an einer bestimmten Position im Dokument.

Da jedes Element (außer dem Dokument selbst) genau ein Elternelement und beliebig viele Kindelemente
besitzt, spricht man auch von dem \emph{DOM-Tree}. Jeder Knoten in diesem Baum besitzt neben den
bereits genannten Methoden und Eigenschaften weitere Attribute, die Informationen über den
jeweiligen Knoten enthalten. Dadurch lässt sich beispielsweise leicht herausfinden ob ein Knoten ein
Element ist, ob es sich um reinen Text handelt oder ob es sich gar um ein weiteres Dokument handelt.

\imga{dom-table.png}{Der DOM-Tree einer HTML-Tabelle \citep{w3c:dom}}{DOM-Tree}






\section{JavaScript}

\subsection{Geschichte}

JavaScript wurde 1995 von Brendan Eich für den \emph{Netscape Navigator 2.0} entwickelt. Die Sprache
war auch in allen darauf folgenden Versionen enthalten und wurde von Microsoft unter dem Namen
\emph{JScript} im \emph{Internet Explorer 3.0} implementiert.

Ab 1996 begann die Standardisierung der Sprache im Rahmen eines ECMA-Standards, welcher im Juni 1997
erschienen ist. Im April 1998 wurde dieser durch die ISO zum internationalen Standard \emph{ISO/IEC
16262} zugelassen.

Im Laufe der Jahre wurde der ECMAScript-Standard immer mehr erweitert und beispielsweise reguläre
Ausdrücke, Exception-Handling und verbesserte Stringfunktionen hinzugefügt. Die aktuelle Ausgabe 5.1
des ECMAScript-Standards entspricht dem internationalen Standard \emph{ISO/IEC 16262:2011}.
\citep{ecmascript}

Auch zum Zeitpunkt der Veröffentlichung dieser Arbeit wird der ECMAScript-Standard und damit die
Sprache JavaScript weiterentwickelt. Jeder moderne Browser unterstützt die Sprache, allerdings
nicht zwangsläufig alle Sprachelemente die in der aktuellsten Spezifikation enthalten sind.


\subsection{Anwendungsgebiete}

Der bekannteste und häufigste Verwendungszweck von JavaScript ist Web-Scripting innerhalb von
Browsern. Dies ist nicht weiter verwunderlich ist, da die Sprache ursprünglich für diesen Zweck
entwickelt und lange Zeit ausschließlich in diesem Kontext benutzt wurde. Dabei stellt der
Webbrowser die Host-Umgebung, also die Objekte und Funktionen zu Kommunikation mit der Applikation,
in der die Scripts ausgeführt werden. Im Webbrowser sind solche Objekte beispielsweise das
Browserfenster (\lstinline{window}), Popupfenster (Rückgabewert von \lstinline{window.open()}),
die Browserhistory (\lstinline{history}), das aktuelle HTML-Dokument (\lstinline{document}).
Ebenfalls vom Browser bereitgestellt werden Objekte, die die einzelnen HTML-Elemente repräsentieren,
also zum Beispiel Hyperlinks, Formulare und Bilder. Darüber hinaus bietet der Browser Methoden, um
über verschiedene Ereignisse wie Klicks, Mausbewegungen und Änderungen an Formularfeldern
benachrichtigt zu werden.
\citep[Kap. 4.1]{ecmascript}

Ein weiteres Anwendungsgebiet ist \emph{Server-Side JavaScript} (\emph{SSJS}). Dabei wird JavaScript
direkt auf den (Web-)Server ausgeführt und kann beispielsweise direkt auf Datenbanken und Dateien
(des Servers) zugreifen. Während \emph{SSJS} erstmalig 1996 im \emph{Netscape Enterprise HTTP Server}
zum Einsatz kam, erlebte es erst fast 15 Jahre später seinen Durchbruch. Mit
\emph{node.js}\footnote{\href{http://nodejs.org}{http://nodejs.org}} existiert ein quelloffenes,
plattformunabhängiges Framework, welches leichtgewichtig ist und dank einer eventbasierten Architektur
auch bei einer sehr großen Anzahl gleichzeitiger Clientverbindungen noch performant ist - eine
Eigenschaft die in klassischen prozess- oder threadbasierten Webservern wie \emph{Apache} oftmals
nicht gegeben ist. Die serverseitige Host-Umgebung unterscheidet sich logischerweise stark von der
eines Browsers, da ein Webserver weder mit Fenstern noch HTML-Dokumenten arbeitet - dafür kennt er
beispielsweise HTTP-Requests und Formular\emph{daten}. Die Ereignisse, auf die Scripte reagieren
können, sind dabei primär netzwerkbezogen - zum Beispiel die neue Verbindung eines Clients oder
der vorzeitige Abbruch eines Seitenaufrufs. Es ist allerdings durchaus möglich, dass mit
entsprechendem Code Teile der Browserumgebung serverseitig nachgebildet werden. So ist es denkbar,
ein HTML-Dokument nicht als String sondern als DOM-Tree darzustellen. In diesem Fall könnten
dieselben Methoden wie im Browser zur Verfügung gestellt werden, sodass Teile des Codes sowohl
client- als auch serverseitig nutzbar sind.


\subsection{Programmierparadigmen}

Das Programmierparadigma einer Programmiersprache ist \enquote{die Sichtweise auf und den Umgang mit
den zu verarbeiteten Daten und Operationen}. \citep[Kap. 1.3.1]{progsprachen} In JavaScript kann man
man sich zwischen drei dieser Paradigmen entscheiden:

\begin{description}
\item[Imperative/prozedurale Programmierung] \hfill \\
Bei der imperativen Programmierung wird eine Folge von Anweisungen sequentiell abgearbeitet. Zur
Kapselung und Wiederverwendung von Funktionalität werden Funktionen genutzt.
\citep[Kap. 1.3.1]{progsprachen}

Da die Host-Umgebung in JavaScript Objekte zur Verfügung stellt und auch Datentypen wie Strings und
Arrays Objekte sind, lässt sich zwar nicht rein prozedural programmieren, jedoch vollständig
auf die Nutzung eigener Objekte verzichten. Da dadurch viel Komfort wie die Nutzung assoziativer
Arrays - die in JavaScript simple Objekte sind - verloren geht, ist dieses Paradigma nicht zu
empfehlen. Es kann aber durchaus sinnvoll sein, Funktionen außerhalb von Objekten zu definieren,
sofern sie alleinstehend sind und man keine Namespaces nutzen will: JavaScript unterstützt zwar
keine Namespace, allerdings sind Funktionen sog. \emph{First-Class-Objekte}, d.h. man kann sie in
Variablen speichern und sie haben noch weitere Eigenschaften, auf die später näher eingegangen
wird. Aufgrund der Möglichkeit, Funktionen in Variablen zu speichern, kann man Namespaces einfach
durch assoziative Arrays simulieren, indem man die Funktionen in solchen Arrays speichert und sie
daher nicht mehr im globalen Kontext liegen sondern nur über das Array aufrufbar sind.

\item[Objektbasierte Programmierung] \hfill \\
Eine objektbasierte Programmiersprache unterstützt Objekte, kennt im Gegensatz zu einer
objekt\emph{orientierten} Programmiersprache allerdings keine Klassen.
\citep[Kap. 1.3.1]{progsprachen}

Nachdem zuvor schon Objekte erwähnt wurden, die von der Host-Umgebung zur Verfügung gestellt werden,
ist es nur logisch, dass der Entwickler auch selbst Objekte definieren kann. Dies geschieht anders
als in klassischen objektorientierten Programmiersprachen wie Java oder C++ nicht über Klassen,
sondern erfolgt hier über Objekte und Konstruktoren (die wiederum ganz normale Funktionen sind).  Da
das Objektsystem von JavaScript einzigartig ist und einige Besonderheiten besitzt, wird im Verlauf
dieses Kapitels näher darauf eingegangen.

\item[Funktionale Programmierung] \hfill \\
Eine rein funktionale Programmiersprache basiert nicht auf Wertzuweisungen sondern benutzt
ausschließlich Funktionsdefinitionen, die eine Eingabe in eine Ausgabe transformieren.
\citep[Kap. 1.3.1]{progsprachen}

JavaScript ist allerdings nicht direkt eine funktionale Programmiersprache und damit erst recht
keine rein funktionale Programmiersprache. Dadurch, dass Funktionen \emph{First-Class-Objekte} sind
und sowohl anonyme Funktionen als auch Closures unterstützt werden, kann man in JavaScript sehr
einfach funktionale Elemente mit objektorientierten bzw. prozeduralen Elementen mischen. Die
prominentesten aus anderen funktionalen Programmiersprachen bekannten Methoden sind
\lstinline{forEach()} um eine Funktion alle Elemente einer Liste anzuwenden, \lstinline{map()} um
eine Liste mittels einer Funktion in eine neue Liste zu transformieren und \lstinline{filter()} um
nur die Elemente aus einer Liste zu übernehmen, die von der Filter-Funktion durch die Rückgabe von
\lstinline{true} akzeptiert wurden.
\end{description}


\subsection{First-Class-Objekte}
In JavaScript handelt es sich bei Funktionen um sogenannte \emph{First-Class-Objekte}. Dies
bedeutet an sich nur, dass es sich bei Funktionen um Objekte handelt - beispielsweise kann eine
Funktion wie jedes andere Objekt Attribute und Methoden (die wiederum Funktionen sind) besitzen.
Details zu den verschiedenen Objekt-Typen einschließlich Funktionsobjekten finden sich in
\citep[Kap. 8.6]{ecmascript}.

\emph{First-Class} beschreibt Eigenschaften die man bei Objekten im Allgemeinen erwartet, im
Zusammenhang mit Funktionen allerdings durchaus ungewöhnlich sind:

\begin{itemize}
\item Sie können in Variablen und anderen Datenstrukturen (beispielsweise Arrays) gespeichert
werden.
\begin{lstlisting}[caption=Zuweisung einer Funktion an eine Variable]
var foo = someFunction;
\end{lstlisting}

\item Sie können Funktionsparameter sein.
\begin{lstlisting}[caption=Erstellen einer neuen Liste mittels einer Transformationsfunktion]
var roots = [4, 9, 16].map(Math.sqrt);
\end{lstlisting}

\item Sie können Rückgabewert einer Funktion sein.

\item Sie können zur Laufzeit erstellt werden.
\begin{lstlisting}[caption=Eine Funktion\, die eine neue Funktion erstellt und zurückgibt]
function makeFunction() {
    function newFunction() {}
    return newFunction;
}
\end{lstlisting}

\item Sie sind nicht an einen Namen gebunden.
\begin{lstlisting}[caption=Erstellen einer anonymen Funktion]
function makeFunction() {
    return function() {};
}
\end{lstlisting}
\end{itemize}


\subsection{Native Objekte}
Wie zuvor erwähnt sind beispielsweise Strings und Arrays Objekte. Diese sind jedoch nicht in
JavaScript selbst implementiert sondern in der Sprache der JavaScript-Engine, also beispielsweise C
oder C++. Ihr Verhalten ist durch die ECMAScript-Spezifikation \citep{ecmascript} vollständig
definiert. Dies bedeutet beispielsweise dass es prinzipiell sicher ist, gewisse Operationen mit
diesen Objekten auszuführen - beispielsweise sie um eigene Methoden zu erweitern. Sofern ein natives
Objekt eine Operation nicht unterstützt wird eine entsprechende Fehlermeldung ausgegeben.



\subsection{Host-Objekte}

Bei Host-Objekten handelt es sich um von der Host-Umgebung zur Verfügung gestellte Objekte die in
der Regel nicht in JavaScript sondern in der Sprache der Host-Umgebung oder der JavaScript-Engine
geschrieben sind. Diese Objekte sind auf die Umgebung zugeschnitten in der das Script läuft - im
Browser werden beispielsweise die DOM-Elemente durch Host-Objekte repräsentiert.
Während native Objekte durch die ECMAScript-Spezifikation definiert sind, gibt es für Host-Objekte
lediglich einige Anforderungen und Einschränkungen, die notwendig sind, damit die JavaScript-Engine
korrekt mit ihnen arbeiten und die Codeausführung optimieren kann: Alle internen Objekteigenschaften
müssen definiert und gewisse Invarianten erfüllt sein.

Das Verhalten dieser Objekte ist jedoch nicht
spezifiziert: \enquote{Host objects may implement these internal methods in any manner unless specified
otherwise} \citep[Kap. 8.6.2]{ecmascript}; die \enquote{internal methods} sind beispielsweise die Getter
und Setter von Objekteigenschaften. Darüber hinaus dürfen auch die Auswirkungen von internen
Eigenschaften erweitert werden: \enquote{Host objects may support these internal properties with any
implementation-dependent behaviour as long as it is consistent with the specific host object
restrictions stated in this document.} \citep[Kap. 8.6.2]{ecmascript}

Ein Beispiel für das undefinierte Verhalten von Host-Objekten findet sich im
Blog\footnote{\href{http://perfectionkills.com/whats-wrong-with-extending-the-dom}{http://perfectionkills.com/whats-wrong-with-extending-the-dom}}
eines ehemaligen Entwicklers eines JavaScript-Frameworks, welches im dritten Kapitel noch näher
betrachtet wird:

\begin{lstlisting}[caption=Verhalten von Host-Objekten im Internet Explorer]
document.createElement('p').offsetParent; // "Unspecified error."
new ActiveXObject("MSXML2.XMLHTTP").send; // "Object doesn't support this property or method"
\end{lstlisting}

In diesem Beispiel wird auf nicht vorhandene Eigenschaften bzw. Methoden von zwei verschiedenen
Host-Objekten lesend zugegriffen. Native und benutzerdefinierte Objekte würden in diesem Fall
einfach \lstinline{undefined} zurückgeben statt eine Exception auszulösen.
Es ist auch durchaus möglich, dass eine Wertzuweisung keine Fehlermeldung bzw. Exception auslöst,
jedoch stillschweigend ignoriert wird und die Eigenschaft des betroffenen Objekts ihren alten Wert
beibehält.


\subsection{Closures}
Normalerweise unterscheidet man in der Programmierung zwischen lokalen Variablen, die nur innerhalb
einer Funktion sichtbar und lebendig sind, globalen Variablen, die überall sichtbar sind und Membervariablen, die
nur über ein Objekt sichtbar sind. Wenn keine Funktionen zur Laufzeit definiert werden, reichen
diese Variablentypen auch aus. Wenn man jedoch Funktionen zur Laufzeit definiert, besteht die
Möglichkeit, dass dies in einer Funktion geschieht, die bereits lokale Variablen besitzt. In der
inneren Funktion sind diese Variablen nun ebenfalls verfügbar. Daher dürfen sie vom Garbage
Collector nicht gelöscht werden obwohl es funktions-lokale Variablen sind und die Funktion, in der
sie definiert wurden, bereits verlassen wurde. Die Kombination aus einer Funktion und ihrer Umgebung
(also den außerhalb definierten Variablen) bezeichnet man als \emph{Closure}. Die innerhalb der
Funktion verfügbaren Variablen, die weder lokal noch global sind, bezeichnet man als
\emph{nichtlokale} Variablen.

Im folgenden Beispiel wird eine Closure genutzt, um eine Funktion zu erstellen, die eine Variable in
ihrer Nutzung insofern einschränkt, dass sie nicht direkt gelesen oder geschrieben werden kann
sondern nur mittels der zurückgegebenen Funktion ausgelesen werden kann, wobei ihr Wert sich bei
jedem dieser Vorgänge um \lstinline{1} erhöht.

\begin{lstlisting}[caption=Beispiel für eine Closure]
function makeIncrementer(val) {
    return function() {
        return ++val;
    }
}
\end{lstlisting}

Eine mögliche Nutzung von Closures ist also die Kapselung von Variablen; ein weiterer
Anwendungsbereich wird später zusammen mit dem Objektmodell von JavaScript vorgestellt.


\subsection{Objekte, Konstruktoren und Prototypen}
Anders als objektorientierte Sprachen wie Java oder C++ ist JavaScript objekt\emph{basiert}. Wie
bereits bei der Vorstellung der Programmierparadigmen erwähnt bedeutet dies, dass JavaScript keine
Klassen besitzt. Um dennoch ein neues Objekt zu erstellen, was in etwa der \emph{Instanz} einer
Klasse entspricht, benutzt man den \lstinline{new}-Operator und eine Konstruktorfunktion
\lstinline{F}. Dieser erstellt ein neues Objekt \lstinline{obj} und weist seiner internen
Eigenschaft \emph{[[Prototype]]} den Wert der Eigenschaft \lstinline{F.prototype} zu sofern er
existiert und ein Objekt ist. Danach wird \lstinline{F} ausgeführt, wobei \lstinline{this === obj}
ist. Der Rückgabewert des \lstinline{new}-Operators ist \lstinline{obj}. \\
Es besteht also eine Möglichkeit, ein neues Objekt zu erstellen und mittels einer Funktion auf
dieses Objekt zuzugreifen bzw. es danach einer Variable zuzuweisen.

Da ein Objekt ohne Methoden allerdings kaum mit einer Klasse vergleichbar ist, muss JavaScript
logischerweise eine Möglichkeit bieten, einem Objekt Methoden zuzuweisen. Eine Möglichkeit wäre, sie
wie in \autoref{lst:func-in-ctor} im Konstruktor dem Objekt hinzuzufügen - dank der
\emph{First-Class}-Eigenschaften von Funktionen ist das auch ohne Weiteres möglich. Allerdings hat
diese Herangehensweise einen entscheidenden Nachteil: Für jedes Objekt wird eine neue Funktion
erstellt - bei vielen Objekten ist diese Lösung also sehr speicherineffizient.

\begin{lstlisting}[label=lst:func-in-ctor,caption=Funktionszuweisung im Konstruktor]
function SomeClass {
    this.doStuff = function() { };
}
\end{lstlisting}

Für die effizientere Lösung ist es wichtig zu wissen, was beim Zugriff auf eine Objekteigenschaft
geschieht: Die JavaScript-Engine führt die interne Methode \emph{[[GetProperty]]} des Objekts aus.
Diese Methode prüft, ob das Objekt selbst die gewünschte Eigenschaft besitzt und gibt sie zurück
falls dies der Fall ist. Ansonsten wird überprüft, ob die \emph{[[Prototype]]}-Eigenschaft des
Objekts \lstinline{null} ist. Falls sie das ist, gibt \emph{[[GetProperty]]} \lstinline{undefined}
zurück. Existiert jedoch ein Prototyp, so wird die \emph{[[GetProperty]]}-Methode des Prototypen
ausgeführt und ihr Rückgabewert zurückgegeben. Es werden also alle Prototypen rekursiv durchlaufen,
bis entweder die Eigenschaft gefunden wurde oder ein Objekt ohne Prototyp erreicht wurde.

Da Methoden ebenfalls Eigenschaften sind wird durch die Nutzung der \emph{prototype chain} sowohl
der erhöhte Speicherverbrauch verhindert - die Funktion ist ja nur noch in einem einzigen Objekt,
dem Prototypen - und außerdem Vererbung und Überschreiben von Funktionen ermöglicht.

\begin{lstlisting}[label=lst:func-in-proto,caption=Funktionszuweisung im Prototypen]
function SomeClass { }
SomeClass.prototype.doStuff = function() { };
\end{lstlisting}

Zur Vererbung weist man dem Prototypen der Konstruktorfunktion einfach ein (neues) Objekt zu,
welches die entsprechenden Funktionen enthält, und fügt die neuen Funktionen danach diesem Objekt
hinzu.

\begin{lstlisting}[label=lst:proto-inheritance,caption=Prototypische Vererbung]
function Parent() { }
Parent.prototype.say = function() { print('Hi'); }
function Child() {
    this.greeting = 'Hallo';
}
Child.prototype = new Parent();
Child.prototype.greet = function() { print(this.greeting); }
c = new Child();
c.greet(); // Hallo
c.say(); // Hi
Child.prototype.say = function() {
    Parent.prototype.say.call(this);
    this.greet();
}
c.say(); // Hi Hallo
\end{lstlisting}

\autoref{lst:proto-inheritance} demonstriert Vererbung und Überschreiben von Funktionen. Bei
\lstinline{this} handelt es sich um eine spezielle Variable, die jeweils auf den aktuellen
Objektkontext zeigt. Dies ist bei Funktionen, die direkt mittels \lstinline{func();} aufgerufen
werden das globale Objekt - im Browser ist dies \lstinline{window}; im Fall eines Methodenaufrufs
über \lstinline{obj.method();} ist \lstinline{this === obj}. Dabei ist zu beachten, dass eine innere
Funktion ein neues \lstinline{this} besitzt; falls das der äußeren Klasse genutzt werden soll, muss
es unter einem anderen Namen in einer Closure gesichert werden. Häufig wird dazu wie in
\autoref{lst:this-self-closure} der Name \lstinline{self} genutzt. Um eine überschriebene Methode
des Elternobjekts aufzurufen, kann man nicht \lstinline{this.method();} nutzen, da
\lstinline{this.method} ja auf die neue Funktion zeigt. Stattdessen muss über
\lstinline{ParentObject.prototype.method} auf die Methode zugreifen. Wenn man diese jedoch direkt
ausführen würde, wäre \lstinline{this === ParentObject.prototype}, was eigentlich nie gewollt ist.
Daher nutzt man die \lstinline{call(thisArg, ...)}-Methode, die jede Funktion besitzt und einem
ermöglicht, den Wert von \lstinline{this} innerhalb der aufgerufenen Funktion manuell festzulegen.

\begin{lstlisting}[label=lst:this-self-closure,caption=Sichern von this mittels einer Closure]
Test.prototype.func = function() {
    var self = this;
    return function() {
        self.doSomething();
    }
};
\end{lstlisting}

Verglichen mit klassenbasierten objektorientierten Sprachen ist das Objektsystem von JavaScript sehr
flexibel, da alle Operationen - vom Hinzufügen von Methoden zu einer "Klasse" bis zum Erstellen einer
neuen "Klasse" - zur Laufzeit möglich sind. Diese Flexibilität hat jedoch den Nachteil, dass
statische Analysen und aus IDEs\footnote{Integrated Development Environment; z.B. Eclipse oder
Visual Studio} bekannte Funktionen wie Autovervollständigung von Methoden und Eigenschaften nicht
immer möglich sind, da der Code für solche Funktionen in der Regel nicht ausgeführt sondern nur
anaylsiert wird.

\subsection{Browserspezifische Unterschiede}\label{js-events}
Aufgrund der verschiedenen Browser-Rendering-Engines, die meist auch unterschiedliche
JavaScript-Engines besitzen, gibt es trotz vorhandenen Standards wie \citep{ecmascript} und
\citep{w3c:dom} Unterschiede, die bei der JavaScript-Programmierung beachtet werden müssen, sofern
man kein Framework nutzt, welches die entsprechenden Operationen abstrahiert bzw. kapselt.
\autoref{lst:js-events-plain} zeigt ein Beispiel für die unterschiedlichen Funktionen, die man nutzen
muss, um auf Events zu reagieren, sofern man auch den Internet Explorer 8 (oder älter) unterstützen
möchte. Wie sich dieselbe Funktionalität durch ein Framework realisieren lässt, ist in
\autoref{lst:js-events-jquery} gezeigt.

\begin{lstlisting}[label=lst:js-events-plain,caption=Event-Handling ohne Abstraktion]
function handleClick(event) {
    if(!event) {
        event = window.event;
    }
    // [...]
}

if(el.addEventListener) {
    el.addEventListener('click', handleClick, false);
}
else if (el.attachEvent) { // IE <=8
    el.attachEvent('onclick', handleClick);
}
\end{lstlisting}

\begin{lstlisting}[label=lst:js-events-jquery,caption=Event-Handling mit jQuery]
$(el).bind('click', function(event) {
    // [...]
});
\end{lstlisting}
