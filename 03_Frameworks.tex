\chapter{JavaScript-Frameworks}

\section{Vergleichskriterien}
Um die Frameworks miteinander vergleichen zu können, müssen einige Kriterien festgelegt werden,
anhand derer sich alle Frameworks messen lassen.

\begin{description}
\item[Zugriff auf Elemente] \hfill \\
JavaScript-Frameworks kapseln den Zugriff auf Elemente in der Regel über Funktionen, die ein oder
mehrere Elemente anhand eines CSS-Selektors suchen und zurückgeben. Dabei können neben den nativ
implementierten CSS-Selektoren auch weitere Selektoren unterstützt werden.

\item[Erstellen und Manipulieren von Elementen] \hfill \\
Häufig möchte man mit JavaScript neue HTML-Elemente erstellen oder vorhandene Elemente verändern.
Dabei kann das Framework den Entwickler unterstützen, indem es beispielsweise das Verknüpfen von
Elementen und die Veränderung von CSS-Attributen vereinfacht.

\item[DOM Traversal] \hfill \\
Unter \emph{DOM Traversal} versteht man das Navigieren durch den DOM-Tree ausgehend von einem
bestimmten Element. Dabei gibt es sowohl einige Standardfunktionen die man in jedem Framework
erwarten kann als auch Komfortfunktionen.

\item[Events] \hfill \\
Wie bereits in \autoref{js-events} gezeigt unterscheiden sich die Eventsysteme der Browser
teilweise. Daher ist dies ein Bereich wo ein Framework sowohl die Browserunterschiede verbergen als
auch den Komfort erhöhen sollte.

\item[Objektsystem] \hfill \\
Manche JavaScript-Frameworks bringen ein eigenes Objektsystem mit, sodass die Arbeit mit Prototypen
und Konstruktoren vereinfacht wird und beispielsweise Mixin-Objekte unterstützt werden oder der
Aufruf einer überschriebenen Method der Parent-Klasse komfortabel möglich ist.

\item[Hilfsfunktionen] \hfill \\
Die meisten JavaScript-Frameworks bieten neben der bereits genannten Funktionalität weitere
Funktionen, die oftmals einen funktionalen Programmierstil vereinfachen oder komfortabler machen.
Allerdings sind nicht nur die funktionalen Funktionen sondern auch jegliche anderen Hilfsfunktionen
betrachtenswert.

\item[UI-Elemente] \hfill \\
Einige JavaScript-Frameworks enthalten UI-Elemente, um beispielsweise Dialoge oder Buttons zu
erzeugen. Die Frameworks unterscheiden sich dort sowohl im Umfang als auch in der Anpassbarkeit.

\item[Kompatibilität] \hfill \\
Bei der Integration in Indico ist es von Vorteil, wenn das Framework möglichst wenig Potenzial
besitzt, Konflikte zu verursachen. Dies kann durch verschiedene Ansätze erreicht werden.

\item[Sonstige Features] \hfill \\
Viele JavaScript-Frameworks haben neben den üblichen Features zusätzliche Funktionen, die in anderen
Frameworks nicht vorhanden sind.

\item[Performance] \hfill \\
Die unterschiedlichen Frameworks sind bei verschiedenen Operationen wie beispielsweise dem Zugriff
auf Elemente anhand eines CSS-Selektors unterschiedlich schnell. Allerdings spielt die Performance
im Rahmen dieser Arbeit nur eine untergeordnete Rolle, da Indico nirgends extrem viele Aktionen auf
einmal ausführt und Performanceunterschiede somit - auch dank den immer schnelleren
JavaScript-Engines - für den Endbenutzer nicht spürbar sind.

\item[Dokumentation] \hfill \\
Da mehrere Entwickler mit dem Framework arbeiten müssen und starke Fluktuation herrscht, da oftmals
Studenten für 3 bis 12 Monate an Indico arbeiten, ist eine gute Dokumentation wichtig, da es nicht
produktiv ist, wenn man erst den Quellcode des Frameworks lesen und verstehen muss um es benutzen zu
können. Insbesondere ist eine Dokumentation hilfreich, wenn sie für alle Funktionen des Frameworks
Beispielcode enthält.

\item[Lizenz] \hfill \\
Die meisten Frameworks sind unter einer OpenSource-Lizenz verfügbar.
Da Indico unter der
GNU~GPL\footnote{\href{http://www.gnu.org/licenses/gpl-2.0.txt}{http://www.gnu.org/licenses/gpl-2.0.txt}}
steht ist auf Kompatibilität mit dieser Lizenz zu achten.
\end{description}



\section{Indico}
Das derzeit in Indico verwendet Framework besteht aus einem Loader-Script und ca. 50
JavaScript-Dateien, die das eigentliche Framework enthalten. In den HTML-Seiten von Indico wird das
Loader-Script jedoch nur im Entwicklungsmodus eingebunden; auf dem Produktivsystem werden die
einzelnen JavaScript-Dateien in einer einzelnen Datei zusammengefasst und komprimiert.

Das Framework ist in mehrere Module aufgeteilt. Im \emph{Core} werden Interfaces,
Iteratorfunktionen, zusätzliche Stringfunktionen und diverse Helferfunktionen implementiert.
Ebenfalls in diesem Modul befindet sich das Objektsystem, welches in Indico genutzt wird.
Das \emph{Data}-Modul enthält ein \emph{Data Binding}-Framework, Funktionen um serverseitige
Aktionen via JSON-RPC\footnote{Remote Procedure Call via AJAX+JSON} auszuführen und diverse
Funktionen zum Verarbeiten von diverser Datentypen wie \lstinline{Date}-Objekten und JSON.

Das größte und im weiteren Verlauf dieser Arbeit wichtigste Modul ist das \emph{UI}-Modul.
Es enthält die in Kapitel 2 erwähnte Kapselung der teilweise browserspezifischen Methoden um auf das
Dokument zuzugreifen und Komfortmethoden zur Erzeugung von DOM-Elementen. Darüberhinaus hat das
Modul verschiedene Submodule: \emph{Draw} abstrahiert das Erstellen von SVG-Grafiken bzw. im
Internet Explorer VRML-Grafiken. \emph{Extensions} und \emph{Styles} erweitern einige Objekte des
\emph{UI}-Moduls um nützliche Methoden. Das \emph{Widgets}-Submodul enthält Funktionen, um
HTML-Elemenet mit häufig benutzen Funktionen bzw. Eventhandlern zu verknüpfen oder mehrere
HTML-Elemente in einer bestimmten Art und Weise zusammenzufügen.
