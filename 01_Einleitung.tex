\chapter{Einleitung}
\pagenumbering{arabic}

JavaScript ist aus einer modernen Webanwendung inzwischen aus vielerlei Gründen nicht mehr
wegzudenken. Es ermöglicht sowohl eine bessere \emph{User Experience} als auch erhöhte Performance.
Beispielsweise kann ein Formular schon vor dem Abschicken an den Server validiert werden; dies gibt
dem Benutzer ein sofortiges Feedback - möglicherweise sogar noch während der Eingabe - und
beansprucht gleichzeitig keine Server-Ressourcen. Darüber hinaus ist es auch möglich Aktionen
auszuführen, ohne dass die komplette Seite neu geladen werden muss. Dies erhöht zum einen das
mögliche Arbeitstempo des Benutzers, da er mehrere Aktionen parallel starten kann, ohne jeweils darauf
warten zu müssen, dass eine Aktion beendet und die Seite neu geladen ist. Zum anderen spart es
Bandbreite, was insbesondere im Zeitalter des mobilen Internets wichtig ist.

Durch die Vielzahl an verschiedenen Browsern (Firefox, Internet Explorer, Chrome, Safari, \ldots) und ihren
unterschiedlichen JavaScript-Engines gibt es allerdings teilweise deutliche Unterschiede darin, wie
man bestimmte Aufgaben mittels JavaScript ausführt. Dazu kommen noch die unterschiedlichen Versionen,
die insbesondere beim Internet Explorer fast schon als verschiedene Browser betrachtet werden
können. Um bei der Entwicklung auf die Eigenheiten der Browser keine Rücksicht nehmen zu müssen,
empfiehlt es sich ein JavaScript-Framework bzw. eine JavaScript-Bibliothek zu nutzen, die die
wichtigsten Funktionen browserunabhängig kapselt und dabei oftmals auch Fehler in Browsern umgeht.
Die meisten JavaScript-Frameworks sind dabei sehr entwicklerfreundlich gehalten, sodass sie auch
unabhängig von der Browserkompatibilität dabei helfen, Zeit und Arbeit bei der Entwicklung
einzusparen.


\section{Zielsetzung}

Die \emph{Indico}-Software nutzt JavaScript um Dialogfenster anzuzeigen, Daten automatisch zu
aktualisieren und Formulare bereits vor dem Abschicken zu validieren. Dabei wird derzeit ein
selbstentwickeltes JavaScript-Framework genutzt, welches größtenteils weder eine Dokumentation
noch eine in sich schlüssige API besitzt. Dieses Framework soll durch ein verbreitetes und gut
dokumentiertes OpenSource-Framework ersetzt werden. Sofern das alte Framework nicht vollständig
ersetzt werden kann, oder der Entwicklungsaufwand dafür zu hoch wäre, ist auf eine möglichst gute
Integration zu achten, sodass kein Framework Konflikte mit dem jeweils anderen verursacht und
bestehender Code weiterhin funktioniert und möglichst wenig angepasst werden muss.

Dabei muss zuerst analysiert werden, welche Funktionalität das bestehende Framework zur Verfügung
stellt und welche Probleme es gibt, damit bei der Wahl des neuen Frameworks insbesondere darauf
geachtet werden kann, dass diese Probleme dort nicht ebenfalls vorhanden sind. Sofern das Framework
User-Interface-Elemente (\enquote{Widgets}) wie beispielsweise Buttons, Dialoge und Tableisten
enthält, ist zu prüfen, ob diese sinnvoll genutzt und an das bestehende Design angepasst werden können.

Nachdem ein Framework ausgewählt wurde, muss dieses in die Software integriert werden. Darauffolgend
müssen möglicherweise auftretende Konflikte mit bestehendem Code behoben werden.
Ebenfalls kann nun analysiert werden, ob es sinnvoll ist, das alte Framework vollständig zu ersetzen
oder nur durch das neue zu erweitern bzw. problematische Teile zu ersetzen und andere zu erweitern
oder überhaupt nicht zu modifizieren.


\section{Aufbau der Arbeit}

Im Einleitungskapitel wird kurz auf die Aufgabenstellung eingegangen und die groben Schritte zum
Ziel beschrieben. Desweiteren wird kurz auf die Firma eingegangen, an der das Projekt durchgeführt
wurde.
Ebenfalls wird die Software, die im Rahmen dieser Arbeit modifiziert wurde, kurz vorgestellt, sodass
man sich einen Überblick darüber verschaffen kann.

Im Grundlagenkapitel werden die genutzen Technologien und ihre Besonderheiten näher betrachtet und
erläutert. Nach einem kurzen Überblick über die Scriptsprache JavaScript wird ihr Objektmodell
vorgestellt und mit dem anderer objektorientierter Sprachen verglichen. Anhand eines Beispiels wird
demonstriert, inwiefern sich die JavaScript-Programmierung für verschiedene Browser unterscheidet,
sofern man kein Framework nutzt.

Auf die Funktionen des bestehenden Frameworks wird im dritten Kapitel kurz eingegangen. Desweiteren
werden dort die in Frage kommenden Frameworks vorgestellt, anhand von verschiedenen Gesichtspunkten
analysiert und mit dem aktuellen Framework verglichen.

Im vierten Kapitel wird die Entscheidung für das jQuery-Framework in Kombination mit der
Underscore.js-Bibliothek begründet, nachdem die Vor- und Nachteile der Migration auf die
verschiedenen Frameworks diskutiert wurden.

Das fünfte Kapitel geht auf die eigentliche Migration und die damit verbundenen Schritte,
Entwicklungen und Probleme ein.

Im sechsten Kapitel wird der aktuelle Stand mit der ursprünglichen Aufgabenstellung verglichen und
kurz auf mögliche zukünftige Erweiterungen eingegangen.


\section{CERN}

Beim CERN (Europäische Organisation für Kernforschung), handelt es sich um eine
Forschungseinrichtung im Schweizer Kanton Genf. Der Hauptaufgabenbereich liegt in der Erforschung
von Grundlagen der Physik, wobei der weltgrößte Teilchenbeschleuniger \emph{LHC} zum Einsatz kommt.

Neben der physikalischen Forschung wird am CERN auch Software entwickelt, die zwar in erster Linie
zur internen Nutzung dient, jedoch oftmals auch als \emph{Open Source} der Allgemeinheit zur Verfügung
gestellt wird. Diese Software erstreckt sich über viele Bereiche der IT, so werden am CERN
beispielsweise Steuersysteme für den Teilchenbeschleuniger, Business-Software für die Verwaltung des
Personals, Grid-Lösungen für die verteilte Datenspeicherung und Webanwendungen für die Archivierung
und Indizierung von Medien entwickelt.

Die Abteilung \emph{IT-UDS-AVC}\footnote{User and Document Services - Audio Visual \& Conference
Rooms} am CERN ist zuständig für die Verwaltung und Einrichtung von video- und telefonbasierten
Konferenzsystemen, Aufzeichnung und Onlinestreaming von Vorträgen, Meetings und Konferenzen und die
Software zur Planung von selbigen. Ebenfalls im Zuständigkeitsbereich der Abteilung sind
die Informationsbildschirme, die in verschiedenen Gebäuden des CERN aufgebaut sind und beispielsweise
den Status des \emph{LHC}\footnote{Large Hadron Collider, der Teilchenbeschleuniger am CERN} und
für die Mitarbeiter relevante Neuigkeiten anzeigen.


\section{Indico}

\subsection{Überblick}
Indico\footnote{Integrated Digital Conference - \href{http://indico-software.org/}{http://indico-software.org/}}
ist eine von der Abteilung \emph{IT-UDS-AVC} am CERN entwickelte Webapplikation, die zum Planen und Organisieren von Events dient. Diese
Events können sowohl einfache Vorträge als auch Meetings und Konferenzen mit mehreren Sessions und
Beiträgen sein. Die Applikation unterstützt außerdem externe Benutzerauthentifizierung, \emph{Paper
Reviewing}\footnote{Im Rahmen eines \emph{Call for Papers} oder \emph{Call for Abstracts} bei einer
Konferenz}, elektronische Sitzungsprotokolle und ein Archiv für die in einer Konferenz benutzten
Materialien. \citep{indico}

Im Laufe der Zeit wurden immer mehr Funktionen hinzugefügt, sodass diese Featureliste schon lange
nicht mehr aktuell ist. Inzwischen enthält Indico beispielsweise ein Buchungs- und
Reservierungssystem für Konferenzzimmer, sodass beim Erstellen eines Events sichergestellt werden
kann, dass der gewünschte Raum auch verfügbar ist und nicht gerade anderweitig benutzt wird.
Ebenfalls in Indico integriert sind die am CERN genutzten Audio- und Videokonferenzsysteme, sodass diese
vollautomatisch eingerichtet und aktiviert werden können sofern diese Systeme im gewählten Raum
verfügbar sind. Eins der neuesten Features ist die Möglichkeit, sich die Konferenzzimmer auf einer
\href{http://maps.google.com/}{\emph{Google Maps}}-basierten Karte anzeigen zu lassen und
beispielsweise anhand der Nähe zu einem bestimmten Gebäude auszuwählen.

\subsection{Aufbau}
Events in Indico sind Teil einer Baumstruktur: Auf der Top-Level-Ebene finden sich ausschließlich
Kategorien, die jeweils wieder Kategorien oder beliebige Events (Konferenzen, Meetings und
Vorträge) enthalten können. Ein Event wiederum kann diverse Elemente enthalten; welche das sind
hängt vom Typ des Events ab, so kann beispielsweise nur eine Konferenz ein Registrierungsformular
besitzen.

\newpage
Die folgende Auflistung beinhaltet die wichtigsten Bestandteile von Events:
\begin{itemize}
\item Audio-/Videokonferenzen
\item \emph{Call for Abstracts}
\item Chaträume
\item Evaluation
\item Materialien
\item \emph{Paper Reviewing}
\item Registrierung
\item Zeitplan
\end{itemize}

\subsection{Architektur}
Der serverseitige Code von Indico ist in Python geschrieben und nutzt
ZODB\footnote{\href{http://www.zodb.org}{http://www.zodb.org}} als Datenbank. Der Code ist in einer
Multi-Tier-Architektur organisiert: Die Anfrage des Webservers wird über
WSGI\footnote{\href{http://www.python.org/dev/peps/pep-0333/}{Web Server Gateway Interface}} an die
Applikation weitergegeben, in der die \emph{Request Handler (RH)}-Ebene die Geschäftlogik ausführt.
Sie prüft die Zugangsberechtigung des Benutzers, verarbeitet GET- bzw. POST-Daten, baut die
Datenbankverbindung auf und führt letztendlich auch die gewünschte Aktion aus. Um eine HTML-Seite
auszugeben, nutzt der \emph{Request Handler} die \emph{Web Pages (WP)}-Ebene. Dort werden diverse
Aktionen ausgeführt, welche die Anzeige der Daten beeinflussen - zum einen werden die benötigten
JavaScript- bzw. CSS-Packages geladen, zum anderen werden, falls vorhanden, Tabs als aktiv markiert
oder Daten sortiert. Die eigentliche Erzeugung der HTML-Datei übernimmt die \emph{Components
(W)}-Ebene. Jede Klasse dieser Ebene repräsentiert ein Template mit demselbem Namen. In der Regel
werden in dieser Ebene nur die aus der WP-Ebene übergebenen Daten an das Template weitergereicht und
falls nötig aufbereitet.

\newpage
Wie \autoref{img.indico-layers.png} verdeutlicht, hat sich Indico im Laufe der Zeit verändert: Statt
dem obsoleten \emph{mod\_python}\footnote{\href{http://www.modpython.org}{http://www.modpython.org}}
wird inzwischen WSGI verwendet.

\img{indico-layers.png}{200px}{Die Architektur von Indico \citep{indicoarch}}{Indico-Architektur}
