\chapter*{Zusammenfassung}
\thispagestyle{empty}

Im Rahmen dieser Bachelor Thesis soll die am CERN entwickelte \emph{Indico}-Software um ein
frei verfügbares JavaScript-Framework erweitert werden. Ein solches Framework erlaubt es,
browserunabhängiger und effizienter zu entwickeln, als es in reinem JavaScript möglich wäre. Bei
Indico handelt es sich um eine Webapplikation zur Planung und Verwaltung von Meetings, Konferenzen
und ähnlichen Events, wobei auch die Verwaltung und Reservierung von Konferenzräumen integriert ist.

Zu Beginn werden die genutzten Technologien HTML und JavaScript vorgestellt. Danach werden sowohl
das derzeitige speziell auf Indico zugeschnittene Framework als auch verschiedene andere Frameworks
vorgestellt und anhand verschiedener Kriterien analysiert. Aufbauend auf diese Analyse werden die
Vor- und Nachteile der Migration zu einem dieser Frameworks untersucht und anhand dieser ein
Framework ausgewählt. Aufbauend auf dem gewählten Framework werden dann Teile von Indico migriert
bzw. angepasst.

Die durch diese Thesis erarbeitete Lösung soll dabei eine Grundlage für die Nutzung von
verbreitetem, gut dokumentiertem \emph{Third Party}-Code und ein wartbares, entwicklerfreundliches
System bieten, welches gleichzeitig auch benutzerfreundlicher als die aktuelle Version ist.
