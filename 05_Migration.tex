\chapter{Migration}

Bei der (teilweisen) Migration zu jQuery sind einige Schritte am Anfang zwingend notwendig, während
andere optional sind und in relativ beliebiger Reihenfolge ausgeführt werden können. Im Folgenden
werden die Migrationsschritte in der Reihenfolge beschrieben, wie sie durchgeführt wurden.

\section{Vorbereitung}

In der Vorbereitungsphase wird jQuery eingebunden und dabei auftretende Konflikte behoben. Es findet
zu diesem Zeitpunkt noch keine Migration statt.

\subsection{Einbinden von jQuery}

Um JavaScript-Dateien in Indico einzubinden fügt man nicht wie üblich
\lstinline[language=HTML]{<script>}-Tags in einem HTML-Template ein sondern erweitert die Funktion
\lstinline{getJSFiles()} der Basisklasse für HTML-Seiten \lstinline{WPBase} sodass sie die
notwendigen Dateien einbindet. Im Falle von jQuery, wo neben jQuery selbst auch jQuery UI und später
diverse Plugins eingebunden werden müssen, bietet sich eine separate Funktion an. Da Indico im
Produktivmodus einzelne JavaScript-Dateien zu komprimierten Paketen \enquote{kompiliert}, muss
zusätzlich zur Python-Funktion die Konfigurationsdatei von \emph{jsmin} angepasst werden, damit die
neuen Dateien in einem solchen Paket zusammengefasst werden.

\begin{lstlisting}[language=Python,caption=Einbinden von jQuery und Underscore.js in Indico]
def _includeJQuery(self):
    info = HelperMaKaCInfo().getMaKaCInfoInstance()
    files = ['underscore', 'jquery', 'jquery-ui']
    if info.isDebugActive():
        return ['js/jquery/%s.js' % f for f in files]
    else:
        return ['js/jquery/jquery.js.pack']
\end{lstlisting}

\subsection{Beheben von Konflikten}

Nachdem jQuery und Underscore.js nun eingebunden sind, muss überprüft werden, ob es Konflikte gibt.
Diese gibt es durch das gleichzeitige Vorhandensein von jQuery und Prototype in der Tat - unabhängig
davon, dass dieser Konflikt bereits aus der Analysephase bekannt ist, zeigt er sich auf Seiten, die
Prototype nutzen - auf diesen funktionieren die JavaScripts nicht mehr korrekt. Der Konflikt
entsteht dadurch, dass jQuery und Prototype beide \lstinline{$} für sich beanspruchen und daher das
zuletzt eingebundene Framework die vorherige Variable überschreibt. Da jQuery nach Prototype
eingebunden wird, ist dieses Problem einfach zu lösen - mittels
\lstinline{var $j = jQuery.noConflict();} wird die Prototype-Funktion wiederhergestellt und jQuery
via \lstinline{$j} verfügbar gemacht. Dabei handelt es sich allerdings um eine temporäre Lösung um
Prototype so lange beizubehalten, bis der darauf basierende Code vollständig migriert wurde. Sie hat
den Vorteil, dass der Prototype-Code teilweise zeilenweise migriert werden und währenddessen immer
wieder auf korrekte Funktion geprüft werden kann.

\subsection{Migration von Prototype-basiertem Code}

Einige Bereiche des \emph{Room Booking}-Systems von Indico nutzen Prototype zum Registrieren von
Events. Darüberhinaus wird Prototype dort zur Formularvalidierung genutzt: Mithilfe des Objekts
\lstinline{Form.Observer} werden die Formulare alle 400ms auf Änderungen überprüft und wenn nötig
erneut validiert. Die häufige Prüfung auf Änderungen ist weder effizient noch notwendig -
Formularelemente können über die \lstinline{change}-, \lstinline{click}- und
\lstinline{keyup}-Events bei jeder Änderung validiert werden ohne mehrmals pro Sekunde alle
Formularelemente auf Änderungen zu überprüfen. Um den Validierungscode mit jQuery möglichst
effizient zu machen werden die Events nicht für alle Formularelemente registriert sondern nur für
das Formular selbst, wobei ein \emph{Delegate} genutzt wird, um speziell auf Events der enthaltenen
Formularelemente zu reagieren. \autoref{lst:jquery-validation-events} zeigt die Registrierung der
Events; im \emph{submit}-Event wird das Formular erneut validiert und, sofern es nicht gültig ist,
das Abschicken des Formulars verhindert und eine Fehlermeldung ausgegeben.

\begin{lstlisting}[caption=Formularvalidierung via jQuery,label=lst:jquery-validation-events]
$j('#bookingForm').delegate(':input', 'keyup change', function() {
    forms_are_valid();
}).submit(function(e) {
    if (!forms_are_valid(true)) {
        e.preventDefault();
        alert($T('There are errors in the form. Please correct the fields with red background.'));
    };
});
\end{lstlisting}

Ebenfalls Änderungsbedarf besteht bei den in \autoref{lst:rb-inline-events} verwendeten
Inline-Eventhandlern. Da die betroffenen Formularelemente bereits IDs besitzen kann ihnen über
jQuery mühelos ein Eventhandler zugewiesen werden - dadurch können alle Scripts an derselben Stelle
im HTML-Code stehen, was den Code lesbarer macht; unabhängig davon muss der Code geändert werden, da
die dort verwendete \lstinline{$}-Funktion aus Prototype stammt.

\begin{lstlisting}[language=HTML,caption=Inline-Eventhandler]
<input id="onlyBookings" type="checkbox" onchange="if (this.checked) $('onlyPrebookings').checked = false;"/>
<input id="onlyPrebookings" type="checkbox" onchange="if (this.checked) $('onlyBookings').checked = false;" />
\end{lstlisting}

An vielen Stellen zahlt es sich aus, dass Prototype und jQuery gewisse Ähnlichkeiten besitzen;
\autoref{lst:prototype-jquery-diff} zeigt die Änderungen eines von Prototype zu jQuery migrierten
Codeblocks.

\begin{lstlisting}[language=diff,label=lst:prototype-jquery-diff,caption=Ähnlichkeit zwischen jQuery und Prototype]
- $('blockedRooms').setValue(Json.write(roomGuids));
+ $j('#blockedRooms').val(Json.write(roomGuids));
\end{lstlisting}
